\section{План решения задачи}
Для решения поставленных задач был составлен следующий план (подробное объяснение плана представлено в секции \ref{5-dataset}).
\begin{enumerate}
    \item Рассмотреть современные подходы для классификации исходного кода и возможность их применения для классификации бинарного кода;
    \item Написать краулер для сбора проектов на языке Rust с платформы Github;
    \item Выбрать сигнатуры функций и комментарии к ним из собранного кода;
    \item \label{class_selection} Выделить классы в предположении, что в реальных проектах полученные данные достаточно хорошо могут описывать семантику функций;
    \item Для полученного размеченного набора данных применить метод классификации, предложенный ранее.
\end{enumerate}

Остановимся подробнее на п.\ref{class_selection} плана.

\subsection{Выделение классов}
Сначала произвести предобработку данных:
\begin{enumerate}
    \item Имена функций, семантически состоящие из нескольких слов, разбить на части перебором по различным стилям написания (camelCase, underscores, PascalCase);
    \item Привести слова к нижнему регистру, удалить все символы-цифры;
    \item Произвести сегментацию полученных слов, так чтобы вероятность полученного разбиения была максимальной;
    \item Произвести лемматизацию;
    \item Удалить слишком короткие, длинные, редкие и частые слова.
\end{enumerate}

Далее полученные токенизированные представления функций перевести в векторы с помощью tf-idf. Затем применить методы для тематической классификации (LSA, PLSA, LDA...) либо использовать алгоритмы кластеризации (k-means, DBSCAN и др.). После этого вручную отобрать полученные классы.