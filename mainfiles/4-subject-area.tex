\section{Описание предметной области}
В дальнейшем тексте будут неоднократно использоваться различные понятия из машинного обучения и внутреннего представления LLVM IR. В этой секции приводится краткое описание предметной области.

\subsection{Связь с обработкой естественных языков}
Рассмотрим сначала, как именно машинное обучение может быть применимо к задаче анализа исходного кода. В области машинного обучения и искусственного интеллекта есть направление ответвление, занимающееся обработкой естественного языка (Natural Language Processing, NLP), которое изучает проблемы компьютерного анализа и синтеза естественных языков. Код на языке программирования является искусственным языком, имеющим строгий синтаксис и семантику, в отличие от естественного языка. Искусственный и естественный языки имеют и другие отличия, однако существует гипотеза, что тексты на языке программирования имеют некоторые статистические особенности, схожие с текстами на естественном языке, и эти особенности могут быть использованы для построения инструментов анализа кода программ. Поэтому существуют методы обработки текстов, которые могут также быть применены к анализу исходного кода программ. 

\subsection{Векторное представление слов}
Во многих методах обработки текстов на естественном языке с помощью машинного обучения используется подход, направленный на получение представления единицы текста (слова, предложения и т.п.) в виде вещественного вектора. Это так называемое векторное представление слов (embedding). Основной целью данного подхода является получение признаков, представляющих информацию об исходных текстах в пригодном для метода формате, т.к. большинство методов машинного обучения принимают на вход в качестве признаков набор вещественных векторов.

\subsection{LLVM}
LLVM\cite{Lattner:MSThesis02} -- проект программной инфраструктуры для создания компиляторов и сопутствующих им утилит. Состоит из набора компиляторов из языков высокого уровня, системы оптимизации, интерпретации и компиляции в машинный код. В основе инфраструктуры используется RISC-подобная платформонезависимая система кодирования машинных инструкций -- байткод LLVM IR, которая представляет собой высокоуровневый ассемблер, с которым работают различные преобразования. Над промежуточным представлением можно производить трансформации во время компиляции, компоновки и выполнения. Из этого представления генерируется оптимизированный машинный код для целого ряда платформ как статически, так и динамически.

Проект LLVM написан на C++ и портирован на большинство Unix-подобных систем и Windows. Система имеет модульную структуру, отдельные её модули могут быть встроены в различные программные комплексы, она может расширяться дополнительными алгоритмами трансформации и кодогенераторами для новых аппаратных платформ.

\subsection{LLVM IR}
LLVM IR -- это промежуточное представление в виде трёхадресного кода в SSA-форме. На практике для хранения кода используется эффективное бинарное представление (bitcode). SSA -- это такая форма промежуточного представления кода, в которой любое значение присваивается только один раз.

LLVM IR поддерживает следующие типы данных:
\begin{itemize}
    \item Целые числа произвольной разрядности. Генерация машинного кода для типов очень большой разрядности не поддерживается, но для промежуточного представления никаких ограничений нет;
    \item Числа с плавающей точкой: float, double, а также ряд типов, специфичных для конкретной платформы (например, x86$\_$fp80);
    \item void -- пустое значение;
    \item Указатели;
    \item Массивы;
    \item Структуры;
    \item Функции;
    \item Векторы.
\end{itemize}

Система типов рекурсивна, поэтому можно использовать многомерные массивы, массивы структур, указатели на структуры и функции, и т.д.

Большинство инструкций в LLVM принимают два аргумента и возвращают одно значение. Значения определяются текстовым идентификатором. Локальные значения обозначаются префиксом $\%$, а глобальные -- @. Локальные значения также называют регистрами, а LLVM -- виртуальной машиной с бесконечным числом регистров.

Тип операндов всегда указывается явно, и однозначно определяет тип результата. Операнды арифметических инструкций должны иметь одинаковый тип, но сами инструкции «перегружены» для любых числовых типов и векторов.
Всего существует 52 инструкции, среди них:
\begin{itemize}
    \item набор арифметических операций;
    \item набор побитовых логических операций и операций сдвига;
    \item специальные инструкции для работы с векторами;
    \item инструкции для обращения к памяти;
    \item операции приведения типа.
\end{itemize}

Из этого краткого обзора видно, что промежуточное представление LLVM достаточно близко соответствует коду на низкоуровневых процедурных языках наподобие Си. При трансляции высокоуровневых языков -- объектно-ориентированных, функциональных, динамических -- придётся выполнить гораздо больше промежуточных преобразований, а также написать специализированный интерпретатор. Но и в этом случае LLVM снимает с разработчика компилятора проблемы кодогенерации для конкретной платформы, берёт на себя большинство независимых от языка оптимизаций и делает их качественно. Кроме этого, с помощью LLVM разработчик получает готовую инфраструктуру для динамической компиляции и возможность оптимизации времени связывания между различными языками, компилируемыми в LLVM.

\subsection{Ida}
IDA Pro Disassembler -- интерактивный дизассемблер, который широко используется для реверс-инжиниринга. Он отличается исключительной гибкостью, наличием встроенного командного языка, поддерживает множество форматов исполняемых файлов для большого числа процессоров и операционных систем. Позволяет строить блок-схемы, изменять названия меток, просматривать локальные процедуры в стеке и многое другое.

IDA до определенной степени может автоматически выполнять анализ кода, используя перекрестные ссылки (xref's), знание параметров вызовов функций стандартных библиотек и другую информацию. Однако одной из самых главных его особенностей считается удобное интерактивное взаимодействие с пользователем. В начале исследования IDA выполняет автоматический анализ программы, а затем пользователь с помощью интерактивных средств начинает давать осмысленные имена, комментировать, создавать сложные структуры данных и другим образом добавлять информацию в листинг, генерируемый дизассемблером, пока не станет ясно, что именно и как делает исследуемая программа.

Дизассемблер имеет консольную и графическую версии. Поддерживает большое количество форматов исполняемых файлов. Одной из отличительных особенностей IDA Pro является возможность дизассемблирования байт-кода виртуальных машин Java и .NET. Также поддерживает макросы, плагины и скрипты, а последние версии содержат интегрированный отладчик.\cite{Ida}

Ida Pro является одним из самых популярных дизассемблеров по нескольким причинам:
\begin{enumerate}
    \item Для Ida существует множество плагинов на языках IDC и Python, например восстановление AST-дерева программы;
    \item Ida использует огромное число эвристик и техник, использующихся при компиляции;
    \item Ida сама комментирует код и распознает стандартные библиотечные и системные функции;
    \item Интерактивность процесса дизассемблирования.
\end{enumerate}

\subsection{Mcsema}
Mcsema\cite{Mcsema} -- это фреймворк, который позволяет переводить бинарный исполняемый файл в llvm биткод.

McSema дает аналитику возможность найти в программе уязвимости, которые очень сложно отследить в исполняемом файле, независимо проверить исходный код поставщика программы и составить тесты с большим покрытием кода. Также оттранслированный LLVM биткод может быть подвержен тестированию с помощью libFuzzer\cite{libFuzzer}. Кроме того, полученный биткод можно скомпилировать в исполняемый бинарный файл.

McSema поддерживает трансляцию ELF и PE форматов исполняемых файлов, а также большинство x86 и amd64 инструкций, включая X87, MMX, SSE и AVX.

Процесс получения llvm биткода состоит из двух шагов: восстановления графа потока управления и трансляции инструкций. Восстановление графа потока управления осуществляется с использованием программы mcsema-disass, которая использует один из следующих дизассемблеров: IDA Pro, Binary Ninja или DynInst. Трансляция инструкций выполняется с помощью программы mcsema-lift, которая переводит граф потока управления в LLVM биткод.
